%页眉页脚设置
\pagestyle{fancy}
\fancyhead{} %clear all fields
\fancyhead[CE]{\songti \zihao{5} 福州大学博士学位论文} %偶数页
\fancyhead[CO]{\songti \zihao{5} 基于在线评论信息的多属性决策方法研究} %奇数页
\fancyfoot[C]{\thepage}%设置页脚
\renewcommand{\headrulewidth}{1pt} %页眉与正文之间的水平线粗细
\renewcommand{\footrulewidth}{0pt}

\phantomsection
\addcontentsline{toc}{section}{第一章~~绪论}
\section*{第一章~~绪论}
\setcounter{section}{1} \setcounter{subsection}{0} 
\setcounter{table}{0} \setcounter{figure}{0} \setcounter{equation}{0} \setcounter{definition}{0}

\subsection{基本使用说明}
文档编译方式:xelatex-biber-xelatex.

每章节前面的图、表、定义、公式的计数器要归0。

图、表、公式使用与引用方式按\LaTeX 规定。

\begin{equation}
    a=b+c \label{eq-1}
\end{equation}

\begin{definition}
    这是一个定义
\end{definition}

参考文献引用\cite{kai1979prospect}。

交叉引用: 公式(\ref{eq-1})。

\subsection{各节一级标题}
我是内容

\subsubsection{各节二级标题}
你是内容


\subsection{字体样式}

{\song 宋体 \kai 楷体 \hei 黑体 \fs 仿宋}

{\tbf 加粗宋体 \kaitib 加粗楷体 \heitib 加粗黑体 \fangsongti 加粗仿宋体}

\newpage