%-------------------设置页眉页脚---------------------
\pagenumbering{Roman} 
\pagestyle{plain}{%
\fancyhf{} % clear all header and footer fields
% \fancyhead[CE]{\song \wuhao 福州大学博士学位论文} %偶数页
% \fancyhead[CO]{\song \wuhao 基于在线评论信息的多属性决策方法研究} %奇数页
\fancyfoot[C]{\thepage}%设置页脚
\renewcommand{\headrulewidth}{0pt}
\renewcommand{\footrulewidth}{0pt}}

%---------------------摘要----------------------------
%中文论文题目:黑体小2,居中,段前17磅,段后30磅,行距20磅
\vspace{17pt}
\begin{center} 
	{\heitib \xiaoerhao
	\setlength{\baselineskip}{20pt} 
	福州大学博士学位论文\LaTeX 模版}
\vspace{30pt}
\end{center}

%摘要标题:居中,4号黑体,段前0行,段后12磅,行距20磅
\begin{center}
\phantomsection
\addcontentsline{toc}{section}{摘~要}  
{\heitib \sihao 摘~要}
\vspace{12pt}
\end{center}

%摘要正文:宋体小四,段前0行,段后0行,行距20磅,首行缩进2字符
\songti \xiaosihao
\setlength{\baselineskip}{20pt} 

本手册假定用户已经能处理一般的\LaTeX 文档,并对其相关知识有一定了解。如果从未接触\LaTeX,建议先学习相关的基础知识。

遵守学术行为规范承诺(即statement每届可能会有不同,请参考当年的毕业论文手册)

如与现行的学校《福州大学学位论文格式要求与规范》不同,以当年发布的《规范》为准。

非官方文档,如有不妥,请指教,请email至shifanhe0828@163.com。

%摘要正文后,中间空一行,另起一行写关键词。
%顶格,小4号黑体,不加粗,中间用逗号(中文)分开,词尾不写标点
\noindent{\heiti \xiaosihao 关键词:关键词1,关键词2,关键词2}

%----------------------英文摘要-------------------------
%英文论文题目:小3 Arial Black,不加粗,居中,段前17磅,段后30磅,行距20磅
\newpage
\begin{center} 
\vspace{17pt}
{\setmainfont{Arial Black} \xiaosanhao Research on Multi-Criteria Decision Making based on Online Reviews}
\vspace{30pt}
\end{center}

%摘要:4号Arial Black,不加粗,居中,段前0行,段后12磅,行距20磅
\begin{center}
\phantomsection
\addcontentsline{toc}{section}{{\large Abstract}}
{\setmainfont{Arial Black} \sihao Abstract} 
\vspace{12pt}
\end{center}


%英文摘要正文,小4 Times New Roman,段前0行,段后0行,行距20磅,首行缩进4个英文字母。
\setmainfont{Times New Roman} \xiaosihao 
\setlength{\baselineskip}{20pt} 

Contents

%英文关键词,中间空一行,另起一行写“key words”。小4 Arial Black,段前0行,段后0行,行距20磅。关键词之间用逗号(英文)隔开,逗号后点击空格键(英文)一次,在写下一个。最后的词尾关键词不写标点。
%在英文关键词中,除人名、地名的首字母,以及简缩写字母要大写外,其余单词的首字母一律小写
\noindent{\setmainfont{Arial Black} \xiaosihao Key words: Online reviews, Heterogeneous information}
%\setmainfont{Times New Roman} 
\newpage

